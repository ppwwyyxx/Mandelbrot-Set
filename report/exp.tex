% File: exp.tex
% Date: Sun Aug 26 21:02:38 2012 +0800
% Author: Yuxin Wu <ppwwyyxxc@gmail.com>
\section{Experience}
	这次使用三种并行库分别进行多线程程序设计,让我有很大进步.
	首先是对于三种并行库都有了更深入的了解.OpenMP程序实现上最容易,但对算法有较大要求,对多线程需要共享信息的情形会难以实现.
Pthread是一个方便的多线程库,编程上难度不大,同时也有\verb|mutex_lock|机制使线程间可以进行信息交流.
MPI由于是多进程的,因此无法共享内存,需要另外使用消息机制来管理通信,因此在实现上较为复杂,需要在原始版本上增添不少代码,效率也不高.
但这也使得MPI程序可以在多个机器上共同运行,适合计算更加巨型的任务.

	上次作业我使用gtkmm创建GUI,这次便改用了Xlib. 相比之下Xlib操作简单,但功能较少.
	尤其是在event处理上只能采用时刻不停接收的方法.如果有多个事件短时间内发生,会一个接一个处理,对性能十分不利.
	例如对于窗口的拉伸操作,就会一次性触发大量\verb|ConfigureNotify|事件,造成程序多次刷新.
	同时,由于多进程/多线程操作内部实现复杂,因此与较复杂的图形库共用时也许会有不可预见的后果,而Xlib的简单结构较易于掌控.

	另外,这次作业生成的Mandelbrot Set,是一个经典的分形图形,将其放大可以找到大量的美丽图片,使我们在完成作业的同时深刻感受到了数学之美.
	
